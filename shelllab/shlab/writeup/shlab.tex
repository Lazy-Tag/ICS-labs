\documentclass[11pt]{article}

\usepackage{times}
\usepackage{alltt}

%% Generic environments and commands that we use in the CS:APP book

%%%%%%%%%% 
% Commands
%%%%%%%%%% 

% For putting labels on pictures
\newcommand{\clabel}[1]{\makebox(0,0){#1}}
\newcommand{\rlabel}[1]{\makebox(0,0)[r]{#1}}
\newcommand{\llabel}[1]{\makebox(0,0)[l]{#1}}
\newcommand{\blabel}[1]{\makebox(0,0)[b]{#1}}
\newcommand{\lblabel}[1]{\makebox(0,0)[lb]{#1}}
\newcommand{\tlabel}[1]{\makebox(0,0)[t]{#1}}

%% Comment command
\newcommand{\comment}[1]{}

%% Two part captions.  First part is title.  Second is explanation.
\newcommand{\mycaption}[2]{\caption[#1]{{\bf #1} #2}}

%%%%%%%%%%%%%%%%%%%%%%%%%%%%%%%%%%%%%%%%%%%%%%%%%%%%%%%%%%%% 
% Environments and macros for different elements of the book
%%%%%%%%%%%%%%%%%%%%%%%%%%%%%%%%%%%%%%%%%%%%%%%%%%%%%%%%%%%%% 

%% Various Unix command-line prompts
\newcommand{\unixprompt}{unix>}
\newcommand{\linuxprompt}{linux>}
\newcommand{\solarisprompt}{solaris>}

%% pcode- Generating pseudo-code
\newenvironment{pcode}{\mbox{}\newline\vspace{.5ex}\newline\begin{minipage}{6in}\begin{tt}\begin{tabbing}MM\=MM\=MM\=MM\=MM\=MM\=MM\=MM\=MM\=\+\kill}{\end{tabbing}\end{tt}\end{minipage}\newline\vspace{.5ex}\newline}

%% Showing syntactic elements in pseudo-code
\newcommand{\syntax}[1]{{\rm\it #1}}

%% aside - for displaying sidebars 
\newenvironment{aside}[1]%
{\begin{quote}\footnotesize{\bf Aside: #1\\}}
{{\bf End Aside.}\end{quote}}

%% ntc - New to C element
\newenvironment{ntc}%
{\begin{quote}\footnotesize{\bf New to C?}\\}
{{\bf End}\end{quote}}

%% ccode- for displaying formatted C code (c2tex) 
\newenvironment{ccode}%
{\small}%
{}

%% scode - for displaying formatted ASM code (s2tex and d2tex)
\newenvironment{scode}%
{\small}
{}

%% codefrag - for displaying unformatted code fragments
\newenvironment{codefrag}%
{\small\begin{alltt}}%
{\end{alltt}%
}

%% tty - for displaying TTY input and output
\newenvironment{tty}%
{\small\begin{alltt}}%
{\end{alltt}}


%% Environment for descriptions that provide aligned text
\newenvironment{mydesc}[1]{
\setbox1=\hbox{#1}
\begin{list}{}{
\setlength{\labelwidth}{\wd1}
\setlength{\leftmargin}{\wd1}
  \addtolength{\leftmargin}{1em}
  \addtolength{\leftmargin}{\labelsep}
\setlength{\rightmargin}{1em}}}{\end{list}}

\newcommand{\litem}[1]{\item[#1\hfill]}
\newcommand{\ritem}[1]{\item[#1]}

%%%%%%%%%%%%%%%%%%%%%%%%%%%%%%%%%%%%%%%%%%%%%%%%%%%%%%%%%%%%%%%%%%%%%%%%%%%%
%% Indexing 
%%%%%%%%%%%%%%%%%%%%%%%%%%%%%%%%%%%%%%%%%%%%%%%%%%%%%%%%%%%%%%%%%%%%%%%%%%%%
%% Distinguish 5 types of index entries
%%   - IA32 instructions (iindex, diindex)
%%   - Program names (pindex, dpindex)
%%   - C things (cindex, dcindex)
%%   - Stdlib functions (sindex, dsindex)
%%   - Unix functions (uindex, duindex)
%%   - CSAPP functions (csindex, dcsindex)
%%   - CSAPP programs (pcsindex, dpcsindex)
%%   - Library files, such as <limits.h> (lindex, dlindex) 
%%   - Everything else. (index, dindex)
%% In every case, the `d' version is used for the definining index entry

%% IA32 instructions.
%% Takes two arguments:
%% The name of the instruction
%% An English rendition of the mnemonic
\newcommand{\iindex}[2]{\index{#1@{\tt #1} [IA32] #2}}
\newcommand{\diindex}[2]{\index{#1@{\tt #1} [IA32] #2|emph}}

%% Program names (e.g., gcc, gdb.  Write in lower case)
%% Takes two arguments:
%% The name of the program
%% An English description
\newcommand{\pindex}[2]{\index{#1@{\sc #1} #2}}
\newcommand{\dpindex}[2]{\index{#1@{\sc #1} #2|emph}}

%% C things (operators, statements, etc)
%% Takes two arguments:
%% C name
%% An English description
\newcommand{\cindex}[2]{\index{#1@{\tt #1} [C] #2}}
\newcommand{\dcindex}[2]{\index{#1@{\tt #1} [C] #2|emph}}

%% Stdlib functions
%% Takes two arguments:
%% function name
%% An English description
\newcommand{\sindex}[2]{\index{#1@{\tt #1} [C Stdlib] #2}}
\newcommand{\dsindex}[2]{\index{#1@{\tt #1} [C Stdlib] #2|emph}}

%% Unix functions
%% Takes two arguments:
%% function name
%% An English description
\newcommand{\uindex}[2]{\index{#1@{\tt #1} [Unix] #2}}
\newcommand{\duindex}[2]{\index{#1@{\tt #1} [Unix] #2|emph}}

%% CSAPP functions
%% Takes two arguments:
%% function name
%% An English description
\newcommand{\csindex}[2]{\index{#1@{\tt #1} [CS:APP] #2}}
\newcommand{\dcsindex}[2]{\index{#1@{\tt #1} [CS:APP] #2|emph}}

%% CSAPP programs
%% Takes two arguments:
%% function name
%% An English description
\newcommand{\pcsindex}[2]{\index{#1@{\sc #1} [CS:APP] #2}}
\newcommand{\dpcsindex}[2]{\index{#1@{\sc #1} [CS:APP] #2|emph}}

%% Library file names (e.g., <limits.h>)
%% Takes two arguments:
%% The name of the file (without brackets)
%% An English description
\newcommand{\lindex}[2]{\index{#1@{\tt <#1>} #2}}
\newcommand{\dlindex}[2]{\index{#1@{\tt <#1>} #2|emph}}

%% Normal index entry.
%% Takes one argument
%% \index already defined
\newcommand{\dindex}[1]{\index{#1|emph}}


%% Page layout
\oddsidemargin 0pt
\evensidemargin 0pt
\textheight 600pt
\textwidth 469pt
\setlength{\parindent}{0em}
\setlength{\parskip}{1ex}

\begin{document}

\title{CS 213, Fall 2002\\
Lab Assignment L5: Writing Your Own Unix Shell\\
Assigned: Oct.~24, Due: Thu., Oct.~31, 11:59PM
}

\author{}
\date{}

\maketitle

Harry Bovik ({\tt bovik@cs.cmu.edu}) is the lead person for
this assignment.

\section*{Introduction}
The purpose of this assignment is to become more familiar with the
concepts of process control and signalling. You'll do this by writing a
simple Unix shell program that supports job control.

\section*{Logistics}

You may work in a group of up to two people in solving the
problems for this assignment.  The only ``hand-in'' will be
electronic.  Any clarifications and revisions to the assignment will
be posted on the course Web page.

\section*{Hand Out Instructions}
\begin{quote}
\bf SITE-SPECIFIC: Insert a paragraph here that explains how the instructor
will hand out the \texttt{shlab-handout.tar} file to the students.
Here are the directions we use at CMU.
\end{quote}

Start by copying the file {\tt shlab-handout.tar}
to the protected directory (the {\em lab directory}) in which you plan
to do your work.  Then do the following:


\begin{itemize}
\item Type the command {\tt tar xvf shlab-handout.tar} 
to expand the tarfile.

\item Type the command {\tt make} to compile and link some
test routines.

\item Type your team member names and Andrew IDs in the header
comment at the top of {\tt tsh.c}.

\end{itemize}

Looking at the {\tt tsh.c} ({\em tiny shell}) file, you will see that
it contains a functional skeleton of a simple Unix shell. To help you
get started, we have already implemented the less interesting
functions.  Your assignment is to complete the remaining empty
functions listed below.  As a sanity check for you, we've listed the approximate
number of lines of code for each of these functions in our reference
solution (which includes lots of comments).
\begin{itemize}
\item {\tt eval}: Main routine that parses and interprets the command
line. [70 lines]
\item {\tt builtin\_cmd}: Recognizes and interprets the built-in
commands: {\tt quit}, {\tt fg}, {\tt bg}, and {\tt jobs}. [25 lines]
\item {\tt do\_bgfg}: Implements the {\tt bg} and {\tt fg} built-in
commands.  [50 lines]
\item {\tt waitfg}: Waits for a foreground job to complete. [20 lines]
\item {\tt sigchld\_handler}: Catches SIGCHILD signals. 80 lines]
\item {\tt sigint\_handler}: Catches SIGINT ({\tt ctrl-c}) signals. [15 lines] 
\item {\tt sigtstp\_handler}: Catches SIGTSTP ({\tt ctrl-z}) signals. [15 lines]
\end{itemize}

Each time you modify your {\tt tsh.c} file, type {\tt make} to
recompile it. To run your shell, type {\tt tsh} to the 
command line:
\begin{tty}
    unix> ./tsh
    tsh>{\em [type commands to your shell here]}
\end{tty}

\section*{General Overview of Unix Shells}

A {\em shell} is an interactive command-line interpreter that runs
programs on behalf of the user. A shell repeatedly prints a prompt,
waits for a {\em command line} on {\tt stdin}, and then carries out
some action, as directed by the contents of the command line.

The command line is a sequence of ASCII text words delimited by
whitespace.  The first word in the command line is either the name of
a built-in command or the pathname of an executable file. The
remaining words are command-line arguments.  If the first word is a
built-in command, the shell immediately executes the command in the
current process.  Otherwise, the word is assumed to be the pathname of
an executable program. In this case, the shell forks a child process,
then loads and runs the program in the context of the child.  The
child processes created as a result of interpreting a single command
line are known collectively as a {\em job}. In general, a job can
consist of multiple child processes connected by Unix pipes.

If the command line ends with an ampersand ''{\tt \&}'', then the job
runs in the {\em background}, which means that the shell does not wait
for the job to terminate before printing the prompt and awaiting the
next command line. Otherwise, the job runs in the {\em foreground},
which means that the shell waits for the job to terminate before
awaiting the next command line. Thus, at any point in time, at most
one job can be running in the foreground. However, an arbitrary number
of jobs can run in the background.

For example, typing the command line 
\begin{tty}
    tsh>{\em jobs}
\end{tty}
causes the shell to execute the built-in {\tt jobs} command. 
Typing the command line
\begin{tty}
    tsh>{\em /bin/ls -l -d}    	
\end{tty}
runs the {\tt ls} program in the foreground.  
By convention, the shell ensures that when the program begins
executing its main routine
\begin{codefrag}
    int main(int argc, char *argv[]) 
\end{codefrag}
the {\tt argc} and {\tt argv} arguments have the following values:
\begin{itemize}
\item {\tt argc == 3},
\item {\tt argv[0] == ``/bin/ls''}, 
\item {\tt argv[1]== ``-l''},
\item {\tt argv[2]== ``-d''}. 
\end{itemize}
Alternatively, typing the command line
\begin{tty}
    tsh>{\em /bin/ls -l -d &}    	
\end{tty}
runs the {\tt ls} program in the background.

Unix shells support the notion of {\em job control}, which allows
users to move jobs back and forth between background and foreground,
and to change the process state (running, stopped, or terminated) of
the processes in a job.  Typing {\tt ctrl-c} causes a SIGINT signal to
be delivered to each process in the foreground job.  The default
action for SIGINT is to terminate the process.  Similarly, typing {\tt
ctrl-z} causes a SIGTSTP signal to be delivered to each process in the
foreground job. The default action for SIGTSTP is to place a process in
the stopped state, where it remains until it is awakened by the receipt
of a SIGCONT signal.  Unix shells also provide various built-in
commands that support job control. For example:
\begin{itemize}
\item {\tt jobs}: List the running and stopped background jobs.
\item {\tt bg <job>}: 
Change a stopped background job to a running background job.
\item {\tt fg <job>}: 
Change a stopped or running background job to a running in the foreground.
\item {\tt kill <job>}: Terminate a job.
\end{itemize}

\section*{The {\tt tsh} Specification}

Your {\tt tsh} shell should have the following features:

\begin{itemize}
\item The prompt should be the string ``{\tt tsh> }''.

\item The command line typed by the user should consist of a {\tt name} and
zero or more arguments, all separated by one or more spaces.  If {\tt
name} is a built-in command, then {\tt tsh} should handle it immediately and
wait for the next command line. Otherwise, {\tt tsh} should assume that
{\tt name} is the path of an executable file, which it loads and runs
in the context of an initial child process (In this context, the term
{\em job} refers to this initial child process).

\item {\tt tsh} need not support pipes ({\tt |}) or
I/O redirection ({\tt <} and {\tt >}). 

\item Typing {\tt ctrl-c} ({\tt ctrl-z}) should cause a SIGINT (SIGTSTP)
signal to be sent to the current foreground job, as well as any
descendents of that job (e.g., any child processes that it forked).
If there is no foreground job, then the signal should have no effect.

\item 
If the command line ends with an ampersand {\tt \&}, then {\tt tsh} should
run the job in the background. Otherwise, it should run the job in the
foreground. 

\item Each job can be identified by either a process ID (PID) or a job
ID (JID), which is a positive integer assigned by {\tt
tsh}. JIDs should be denoted on the command line by the prefix
'\verb:%:'. For example, ``{\tt \%5}'' denotes JID 5, and ``{\tt 5}''
denotes PID 5. (We have provided you with all of the routines you need
for manipulating the job list.)

\item {\tt tsh} should support the following built-in commands:
\begin{itemize}
\item 
The {\tt quit} command terminates the shell.

\item 
The {\tt jobs} command lists all background jobs. 

\item 
The {\tt bg <job>} command restarts {\tt <job>} by sending it a
SIGCONT signal, and then runs it in the background. The {\tt <job>}
argument can be either a PID or a JID. 

\item 
The {\tt fg <job>} command restarts {\tt <job>} by sending it
a SIGCONT signal, and then runs it in the foreground. The {\tt <job>} 
argument can be either a PID or a JID. 
\end{itemize}

\item {\tt tsh} should reap all of its zombie children.  If any job
terminates because it receives a signal that it didn't catch, then
{\tt tsh} should recognize this event and print a message with the
job's PID and a description of the offending signal.

\end{itemize}

\section*{Checking Your Work}

We have provided some tools to help you check your work.

{\bf Reference solution.} The Linux executable {\tt tshref} is the
reference solution for the shell. Run this program to resolve any
questions you have about how your shell should behave. {\em Your shell
should emit output that is identical to the reference solution} 
(except for PIDs, of course, which change from run to run).

{\bf Shell driver.} The {\tt sdriver.pl} program executes a shell as a
child process, sends it commands and signals as directed by a {\em
trace file}, and captures and displays the output from the shell.

Use the -h argument to find out the usage of {\tt sdriver.pl}:
\begin{tty}
unix> ./sdriver.pl -h
Usage: sdriver.pl [-hv] -t <trace> -s <shellprog> -a <args>
Options:
  -h            Print this message
  -v            Be more verbose
  -t <trace>    Trace file
  -s <shell>    Shell program to test
  -a <args>     Shell arguments
  -g            Generate output for autograder
\end{tty}

We have also provided 16 trace files ({\tt
trace\{01-16\}.txt}) that you will use in conjunction with the shell
driver to test the correctness of your shell. The lower-numbered trace
files do very simple tests, and the higher-numbered tests do more
complicated tests.

You can run the shell driver on your shell using trace file 
{\tt trace01.txt} (for instance) by typing:
\begin{tty}
unix>{\em ./sdriver.pl -t trace01.txt -s ./tsh -a "-p"}
\end{tty}
(the {\tt -a "-p"} argument tells your shell not to emit a prompt), or
\begin{tty}
unix>{\em make test01}
\end{tty}
Similarly, to compare your result with the reference shell, you can run 
the trace driver on the reference shell by
typing:
\begin{tty}
unix>{\em ./sdriver.pl -t trace01.txt -s ./tshref -a "-p"}
\end{tty}
or
\begin{tty}
unix>{\em make rtest01}
\end{tty}
For your reference, {\tt tshref.out} gives the output of the reference
solution on all races. This might be more convenient for you than
manually running the shell driver on all trace files.

The neat thing about the trace files is that they generate the same
output you would have gotten had you run your shell interactively
(except for an initial comment that identifies the trace).  For
example:

\begin{tty}
bass> make test15
./sdriver.pl -t trace15.txt -s ./tsh -a "-p"
#
# trace15.txt - Putting it all together
#
tsh> ./bogus
./bogus: Command not found.
tsh> ./myspin 10
Job (9721) terminated by signal 2
tsh> ./myspin 3 &
[1] (9723) ./myspin 3 &
tsh> ./myspin 4 &
[2] (9725) ./myspin 4 &
tsh> jobs
[1] (9723) Running    ./myspin 3 &
[2] (9725) Running    ./myspin 4 &
tsh> fg %1
Job [1] (9723) stopped by signal 20
tsh> jobs
[1] (9723) Stopped    ./myspin 3 &
[2] (9725) Running    ./myspin 4 &
tsh> bg %3
%3: No such job
tsh> bg %1
[1] (9723) ./myspin 3 &
tsh> jobs
[1] (9723) Running    ./myspin 3 &
[2] (9725) Running    ./myspin 4 &
tsh> fg %1
tsh> quit
bass> 
\end{tty}


\section*{Hints}

\begin{itemize}
\item Read every word of Chapter 8 (Exceptional Control Flow)
in your textbook.

\item Use the trace files to guide the development of your
shell. Starting with {\tt trace01.txt}, make sure that your shell
produces the {\em identical} output as the reference shell. Then move
on to trace file {\tt trace02.txt}, and so on.

\item The {\tt waitpid}, {\tt kill}, {\tt
fork}, {\tt execve}, {\tt setpgid}, and {\tt sigprocmask} functions
will come in very handy. The WUNTRACED and WNOHANG options to {\tt
waitpid} will also be useful.

\item When you implement your signal handlers, be sure to send
{\tt SIGINT} and {\tt SIGTSTP} signals to the entire foreground
process group, using ''{\tt -pid}'' instead of ''{\tt pid}'' in the
argument to the {\tt kill} function. The {\tt sdriver.pl} program tests
for this error.

\item One of the tricky parts of the assignment is deciding on the allocation
of work between the {\tt waitfg} and {\tt sigchld\_handler} functions.
We recommend the following approach:
\begin{itemize}
\item In {\tt waitfg}, use a busy loop around the {\tt sleep} function.  

\item In {\tt sigchld\_handler}, use exactly one call to {\tt waitpid}.  
\end{itemize}

While other solutions are possible, such as calling {\tt waitpid} in
both {\tt waitfg} and {\tt sigchld\_handler}, these can be very
confusing. It is simpler to do all reaping in the handler.

\item In {\tt eval}, the parent must use {\tt sigprocmask} to 
block {\tt SIGCHLD} signals before it forks the child, and then
unblock these signals, again using {\tt sigprocmask} after it adds the
child to the job list by calling {\tt addjob}. Since children inherit
the {\tt blocked} vectors of their parents, the child must be sure to
then unblock {\tt SIGCHLD} signals before it execs the new program.

The parent needs to block the {\tt SIGCHLD} signals in this way in
order to avoid the race condition where the child is reaped by {\tt
sigchld\_handler} (and thus removed from the job list) {\em before} the
parent calls {\tt addjob}.
 
\item Programs such as {\tt more}, {\tt less}, {\tt vi}, and {\tt emacs}
do strange things with the terminal settings. Don't run these programs
from your shell. Stick with simple text-based programs such as {\tt
/bin/ls}, {\tt /bin/ps}, and {\tt /bin/echo}.

\item When you run your shell from the standard Unix shell, your shell
is running in the foreground process group. If your shell then creates
a child process, by default that child will also be a member of the
foreground process group. Since typing {\tt ctrl-c} sends a SIGINT to every
process in the foreground group, typing {\tt ctrl-c} will send a SIGINT to
your shell, as well as to every process that your shell created, which
obviously isn't correct.

Here is the workaround: After the {\tt fork}, but before the {\tt
execve}, the child process should call {\tt setpgid(0, 0)}, which puts
the child in a new process group whose group ID is identical to the
child's PID. This ensures that there will be only one process, your
shell, in the foreground process group. When you type {\tt ctrl-c}, the
shell should catch the resulting SIGINT and then forward it to the
appropriate foreground job (or more precisely, the process group that
contains the foreground job).
\end{itemize}

\section*{Evaluation}

Your score will be computed out of a maximum of 90 points based on the
following distribution:
\begin{description}
\item[80] Correctness: 16 trace files at 5 points each.
\item[10] Style points. We expect you to have good comments (5 pts)
and to check the return value of EVERY system call (5 pts).
\end{description}

Your solution shell will be tested for correctness on a Linux machine,
using the same shell driver and trace files that were included in your
lab directory. Your shell should produce {\bf identical} output on
these traces as the reference shell, with only two exceptions:
\begin{itemize}
\item The PIDs can (and will) be different.
\item The output of the {\tt /bin/ps} commands in {\tt trace11.txt},
{\tt trace12.txt}, and {\tt trace13.txt} will be different from run to
run.  However, the running states of any {\tt mysplit} processes in
the output of the {\tt /bin/ps} command should be identical.
\end{itemize}

\section*{Hand In Instructions}

\begin{quote}
\bf SITE-SPECIFIC: Insert a paragraph here that explains how the students
should hand in their \texttt{tsh.c} files.
Here are the directions we use at CMU.
\end{quote}

\begin{itemize}
\item Make sure you have included your names and Andrew IDs in the
header comment of {\tt tsh.c}.

\item Create a team name of the form:
\begin{itemize}
\item ``${\it ID}$'' where ${\it ID}$ is your Andrew ID, if you are
working alone, or
\item ``${\it ID}_1${\tt +}${\it ID}_2$'' where ${\it ID}_1$ is the
Andrew ID of the first team member and ${\it ID}_2$ is the Andrew ID
of the second team member.  
\end{itemize}
We need you to create your team names in this way so that we can
autograde your assignments.

\item To hand in your {\tt tsh.c} file, type:
\begin{tty}
make handin TEAM=teamname
\end{tty}
where \verb|teamname| is the team name described above.

\item After the handin, if you discover a mistake and want to
submit a revised copy, type
\begin{tty}
make handin TEAM=teamname VERSION=2
\end{tty}
Keep incrementing the version number with each submission.

\item You should verify your handin by looking in 
\begin{tty}
/afs/cs.cmu.edu/academic/class/15213-f01/L5/handin
\end{tty}
You have list and insert permissions in this directory, but no
read or write permissions.
\end{itemize}
Good luck! 


\end{document}
